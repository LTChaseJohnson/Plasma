\documentclass{article}
\usepackage{graphicx}

\begin{document}

\title{MAGNETIC RECONNECTION IN SPACE AND LABORATORY PLASMAS AND IT’S IMPLICATION IN TOKAMAK PLASMA CONFINEMENT}
\author{Chase Johnson}

\maketitle

\begin{abstract}
Magnetic reconnection is the topological restructuring of magnetic field lines in high temperature plasmas.    The aim of this paper will be to describe the fundamental physics of magnetic reconnection and discuss the various MHD models to describe magnetic reconnection.  Magnetic reconnection as observed in both space plasmas and laboratory plasmas results in significant changes in kinetic and thermal properties of the plasma.  This inherently deteriorates the ability to confine plasma in the laboratory.  The second part of this paper will describe the fundamentals of saw-tooth instabilities triggering magnetic reconnection in tokamak plasma devices.  Due to the undesirable effects of magnetic reconnection in laboratory plasmas, several techniques have been developed to control sawtooth instabilities.  The efficacy of these techniques to include current drive schemes and ion population control will be discussed.
\end{abstract}
\section{Reconnection Defined}
Reconnection is the rearranging of the magnetic topology where magnetic field lines are broken and then recombine and is shown in Figure 1.$^{10,13,14}$  This is an important process to understand in astrophysics, laboratory plasma science, and others due to it's disruptive effects.  The process of reconnection changes the macroscopic quantities of plasmas through:$^{10}$
\begin{itemize}
\item conversion of magnetic energy into heat
\item accelerate plasma by converting magnetic energy into kinetic energy
\item create shockwaves, current filamentation, and turbulence
\item affect fluxes of fast particles and heat due to changes in global magnetic field lines
\end{itemize}
The first two effects are especially important for stability of confined plasmas due to causing major and minor disruptions in tokomak discharges that will be discussed in this paper.$^{14}$
\begin{figure}[h]
\centering
\includegraphics[width=0.7\linewidth]{"../../../Dropbox/Graduate_School/Plasma/Plasma References/Dendy1"}
\caption{An intial perturbation generates magnetic pressure to counter. However, magnetic field diffusion occurs rapidly in locations of strong current. As magnetic field diffuses, perturbation is allowed to grow and alter topology.$^{5}$}
\label{fig:Dendy1}
\end{figure}
\subsection{Reconnection in Space Plasmas}
Magnetic reconnection is seen directly in many aspects of space plasmas.  Solar flares and coronal mass ejections are one of the first events studied to understand magnetic reconnection.$^{13}$  Coronal mass ejections and solar flares result form converting magnetic energy into heat and particle energy during a magnetic reconnection event.$^{10}$  Additionally, a small scale reconnection model is thought to be responsible for the higher density slow solar wind originating from the Sun's equatorial region.$^{10}$  Reconnection between solar wind magnetic field lines and the Earth's magnetoshpere field lines is also thought to be responsible for the penetration of solar particles to penetrate the magnetosphere(Figure 2).$^{6,14}$\\\\\begin{figure}
\centering
\includegraphics[width=0.7\linewidth]{"../../../Dropbox/Graduate_School/Plasma/Plasma References/Nature1"}
\caption{Reconnection at magnetopause$^{6}$}
\label{fig:Nature1}
\end{figure}

Understanding reconnection in space plasma is vitally important to protecting
\subsection{Why is Reconnection Bad}
\section{References}
\begin{enumerate}
\item Aymar, R. \textit{et al} "The ITER design" 2001 \textit{Plasma Phys. Control Fusion} \textbf{44} 519.
\item Chapman, I. T. "Controlling sawtooth oscillations in tokamak plasmas" 2011 \textit{Plasma Phys. Control Fusion} \textbf{53} 1.
\item Chapman, I. T. \textit{et al} "Magnetic Reconnection Triggering Magnetohydrodynamic Instabilities during a Sawtooth Crash in a Tokamak Plasma" 2010 \textit{Phys. Rev. Lett.} \textbf{105} 255002.
\item Chapman, I. T. \textit{et al} "The physics of sawtooth stabilization" 2007 \textit{Plasma Phys. Control Fusion} \textbf{49} 12B.
\item Hood, A. W. "Solar plasmas" In Dendy, Richard (Ed.), \underline{Plasma Physics: An Introductory Course} (pp. 267-289). Cambridge University Press, 1993.
\item Frey, H. U. \textit{et al} "Continuous magnetic reconnection at Earth's magnetopause" 2003 \textit{Letters to Nature} \textbf{426}.
\item Fridman, Alexander and Lawrence A. Kennedy,  \underline{Plasma Physics and Engineering},  Taylor and Francis Books, Inc., 2004.
\item Harry, John Earnest,  \underline{Introduction to Plasma Technology}, Wiley-VCH, 2010.
\item Hastie, R. J. "Sawtooth Instability in Tokamak Plasmas" 1998 \textit{Astrophysics and Space Science} \textbf{256} 177-204.
\item Priest, Eric,  \underline{Magnetohydrodynamics of the Sun},  Cambridge University Press, 2014.
\item Wagner, F. "A quarter-century of H-mode studies" 2007 \textit{Plasma Phys. Control. Fusion} \textbf{49} B12.
\item Wagner, F. "The Physics Basis of ITER Confinement" 2009 \textit{AIP Conf. Proc.} \textbf{1095} 31.
\item Yamada, Masaaki \textit{et al} "Investigation of magnetic reconnection during a sawtooth crash in a high‐temperature tokamak plasma" 1994 \textit{Phys. Plasmas} \textbf{1} 3269.
\item Yamada, Masaaki \textit{et al} "Magnetic Reconnection" 2010 \textit{Rev. Mod. Phys.} \textbf{82} 603.
\item Yun, G. S. \textit{et al} "Appearance and Dynamics of Helical Flux Tubes under Electron Cyclotron Resonance Heating in the Core of KSTAR Plasmas" 2012 \textit{Phys. Rev. Lett.} \textbf{109} 145003.
\end{enumerate}

\end{document}